\title{MECH 493 - Script for Angular Actuation Case}
\author{Michael Sleeman \\
31150139}

\documentclass[12pt,letterpaper,titlepage]{article}
\usepackage{amsmath}
\usepackage{graphicx}
\usepackage[letterpaper, portrait, margin=1in]{geometry}
\usepackage[
open,
openlevel=2,
atend,
numbered
]{bookmark}

\newcommand{\uvec}[1]{\mathbf{\hat{#1}}}
\newcommand{\gvec}[1]{\mathbf{\underline{#1}}}

\begin{document}
\maketitle
\section{PROBLEM FORMULATION}
\subsection{Statements and Assumptions}
The system is boundary actuated, where the angle that the left most filament ($i=1$) makes with the horizontal axis varies sinusoidally with time. The left end of the left most filament is fixed in space and its angle is adjusted so that:
\begin{equation}
x_1 = 0, \ y_1 = 0, \ \theta_1 = \Theta \cos \omega t
\end{equation}

To find the hydrodynamic forces we must know $\gamma$, $l$, $\theta_i$, $\dot{x}_i$, $\dot{y}_i$, and $\dot{\theta}_i$. To find the hydrodynamic torque we must know all of the above, as well as $x_i$ and $y_i$.

\subsection{Initial Conditions}
At the start of the simulation, we will assume that the filament is at rest and that all angles are $\theta_i = \Theta$. The kinematic constraints must also be satisfied so that
\begin{align*}
x_{i+1} &= x_{i} + l \cos \Theta \\
y_{i+1} &= y_{i} + l \sin \Theta 
\end{align*}

We can set up a system of equations to solve for the linear and angular velocities of each of the filaments. Using the initial conditions, we can integrate in time to solve for the position and angle of each of the filaments at a given time step.

\subsection{Knowns and Unknowns}
As stated above, we know the initial conditions for each of the filaments. We can solve a system of equations involving force balances, torque balances, and kinematic constraints to determine the linear and angular velocities of each of the filaments. Integrating in time will give us the positions and angles of each of the filaments at the next time step. The process is as follows:
\begin{enumerate} \itemsep -2pt
\item Use initial conditions to create a system of equations
\item Solve system of equation to find linear and angular velocities of each filament
\item Integrate in time to find positions of each filament for the next time step
\item Use positions of filaments in current time step to create a system of equations
\item Solve system of equations to find linear and angular velocities of each filament
\item Integrate in time to find positions of each filament for the next time step.
\item Repeat steps 4 through 6 as many times as necessary.
\end{enumerate}

\newpage

We also have more information about the first filament. We know that the angle of the first filament is actuated such that $\theta_1 = \Theta \cos \omega t$. We can find the filaments angular velocity by taking the derivative of its angle to find $\dot{\theta}_1 = - \Theta \omega \sin \omega t$. We also assume that the first filament is fixed in space. This gives three knowns for the first filament. We assume that all of the filaments are rigid and inextensible, therefore the position of the second filament is known at all points in time, as follows:

\begin{align*}
x_2 = l \cos(\theta_1)\\
y_2 = l \sin(\theta_1)\\
\end{align*}

However, the angle that the second filament makes with the horizontal axis is unknown. This gives two knowns for the second filament. In total there are five knowns. We need to know $\dot{x}_i$, $\dot{y}_i$, and $\dot{\theta}_i$ for each of the filaments $1$ through $N$. This gives $3N$ unknowns. We also need to know that linear velocities of the end of filament $N$, at position $N+1$. This gives two more unknowns. Summing all of the knowns and unknowns gives $3N + 2 - 5 = 3N-3 = 3(N-1)$ unknowns. We will need an equal number of equations to solve the problem.

\subsection{Force Balance}
The force balance does not give any equations to solve the system. We only know that the actuation force applied to the first filament must balance the hydrodynamic forces generated by the filaments. This information allows us to solve for the thrust generated by the swimmer. We are interested in the thrust in the $x$ direction.

We know that $\dot{x}_1 = \dot{y}_1 = 0$. Therefore $F_{1,x} = (F_{1,x,\dot{\theta}}) \dot{\theta}_1 = (\frac{1}{2} ( \frac{1}{N} )^2 \sin \theta_1) \dot{\theta}_1$. We also know that $\dot{\theta}_2$ is not equal to zero at all points in time, as well as the formulas for $\dot{x}_2$ and $\dot{y}_2$.

\subsection{Torque Balance}
The end of filament $N$ is free, so the hydrodynamic torque acting on this filament should balance the torque in the spring at $\mathbf{x}_N$. Taking the moment sum about point $\mathbf{x}_N$ gives
\begin{equation}
\uvec{e}_z  \cdot \mathbf{T}_{N,N}^h - k_N(\theta_N - \theta_{N-1}) = 0
\end{equation}

Similarly, the hydrodynamic torque acting on filaments $N$ and $N-1$ should balance the torque in the spring at $\mathbf{x}_{N-1}$. Taking the moment sum about point $\mathbf{x}_{N-1}$ gives
\begin{equation}
\uvec{e}_z \cdot (\mathbf{T}_{N-1,N-1}^h + \mathbf{T}_{N,N-1}^h) - k_{N-1}(\theta_{N-1} - \theta_{N-2}) = 0
\end{equation}

This reasoning can be extended to create $N-1$ torque balance equations where
\begin{align}
\begin{split}
\uvec{e}_z &\cdot \sum_{i=2}^{N} \mathbf{T}_{i,2}^h - k_2(\theta_2 - \theta_1) = 0 \\
\uvec{e}_z &\cdot \sum_{i=3}^{N} \mathbf{T}_{i,3}^h - k_3(\theta_3 - \theta_2) = 0 \\
&\vdots \\
\uvec{e}_z &\cdot \mathbf{T}_{N,N}^h - k_N(\theta_N - \theta_{N-1}) = 0
\end{split}
\end{align}

\newpage

\subsection{Spring Constant}
For a cantilever beam, the deflection at the end of the beam is given by $x = \frac{F}{k}$, where $k$ is the spring constant given by
\begin{equation}
k = \frac{3EI}{l^3}
\end{equation}

We can use conservation of potential energy to find the equivalent torsional spring constant $k_t$:
\begin{align}
\begin{split}
\frac{1}{2} k_t \theta^2 &= \frac{1}{2} k x^2 = \frac{1}{2} k (l \theta)^2 \\
k_t &= k (\frac{l \theta}{\theta})^2 = k l^2 = \frac{3EI}{l^3} l^2 = \frac{3EI}{l}\\
\end{split}
\end{align}

We can create a new variable $A_0 = 3EI$, where $A_0$ is a characteristic bending stiffness. This allows us to write the equivalent torsional spring constant of the beam as
\begin{equation}
k_t = \frac{A_0}{l}
\end{equation}

\subsection{Stiffness Distribution}
We will examine linear, quadratic, and exponentional stiffness distributions. The stiffness distributions are given by the following equations
\begin{align}
\begin{split}
\text{Linear: } &= as + b \\
\text{Quadratic: } &= as^2 + bs + c \\
\text{Exponentional: } &= e^{as}
\end{split}
\end{align}

\subsection{Kinematic Constraints}
The kinematic constraints are:
\begin{align}
\begin{split}
& {x}_i - {x}_{i+1} + l \cos \theta_i = 0 \\
& {y}_i - {y}_{i+1} + l \sin \theta_i = 0
\end{split}
\end{align}

Differentiation of the kinematic constraints gives
\begin{align} \label{eqn:diffKinConstraint}
\begin{split}
& \dot{x}_i - \dot{x}_{i+1} - l \dot{\theta}_i \sin \theta_i = 0 \\
& \dot{y}_i - \dot{y}_{i+1} + l \dot{\theta}_i \cos \theta_i = 0
\end{split}
\end{align}

We know $\theta_1 = \Theta \cos \omega t$ and $\dot{\theta_1} = - \Theta \omega \sin \omega t$ at all points in time. We also know that the left end of the first filament is fixed in space so that $x_1 = y_1 = \dot{x}_1 = \dot{y}_1 = 0$. Therefore the following is true at all points in time:
\begin{align} \label{eqn:xydot2}
\begin{split}
\dot{x}_{2} &= \dot{x}_1 - l \dot{\theta}_1 \sin \theta_1 = 0 - l (- \Theta \omega \sin \omega t) \sin \theta_1 \\
&= l \Theta \omega \sin \omega t \sin \theta_1\\
\dot{y}_{2} &= \dot{y}_1 + l \dot{\theta}_1 \cos \theta_1 = 0 + l (- \Theta \omega \sin \omega t) \cos \theta_1\\
&= - l \Theta \omega \sin \omega t \cos \theta_1
\end{split}
\end{align}

Using the fact that $\dot{x}_{2} = l \Theta \omega \sin \omega t \sin \theta_1$, $\dot{y}_{2} = - l \Theta \omega \sin \omega t \cos \theta_1$, and the kinematic constraints given in (\ref{eqn:diffKinConstraint}), we can find the two following equations:
\begin{align}
\begin{split}
\dot{x}_{2} - \dot{x}_{3} - l \dot{\theta}_2 \sin \theta_2 &= 0\\
l \Theta \omega \sin \omega t \sin \theta_1 - \dot{x}_{3} - l \dot{\theta}_2 \sin \theta_2 &= 0\\
- \dot{x}_{3} - l \dot{\theta}_2 \sin \theta_2 &= - l \Theta \omega \sin \omega t \sin \theta_1
\end{split}
\end{align}

\begin{align}
\begin{split}
\dot{y}_{2} - \dot{y}_{3} + l \dot{\theta}_2 \cos \theta_2 &= 0\\
- l \Theta \omega \sin \omega t \cos \theta_1 - \dot{y}_{3} + l \dot{\theta}_2 \cos \theta_2 &= 0\\
- \dot{y}_{3} + l \dot{\theta}_2 \cos \theta_2 &= l \Theta \omega \sin \omega t \cos \theta_1
\end{split}
\end{align}

We can then use the kinematic constraints given in (\ref{eqn:diffKinConstraint}) to find $2(N-2) = 2N - 4$ equations in the $x$ and $y$ axes
\begin{align}
\begin{split}
\dot{x}_{i} - \dot{x}_{i+1} - l \dot{\theta}_i \sin \theta_i &= 0 \text{, for } i = 3,4,...,N \\
\dot{y}_{i} - \dot{y}_{i+1} + l \dot{\theta}_i \cos \theta_i &= 0 \text{, for } i = 3,4,...,N
\end{split}
\end{align}

This gives a total of $2N-4 + 2 = 2N - 2 = 2(N-1)$ equations for the kinematic constraints. We know that we have $3N-3$ unknowns. We now have $2N - 2 + N - 1= 3N - 3$ equations, and our system is solvable.

\newpage
\section{NON-DIMENSIONAL EQUATIONS}
\subsection{Non-Dimensional Variables}
We can non-dimensionalize the variables $t$, $x_i$, $y_i$, $\dot{x}_i$, $\dot{y}_i$, and $\dot{\theta}_i$, using $L$ for the length scale, $\omega$ for the time scale, and $L \omega$ for the velocity scale. This gives

\begin{align}
\begin{split}
t^* &= \omega t \\
x_i^*&= \frac{x_i}{L} \\
y_i^* &= \frac{y_i}{L} \\
\dot{x}_i^* &= \frac{\dot{x}_i}{L \omega} \\
\dot{y}_i^* &= \frac{\dot{y}_i}{L \omega} \\
\dot{\theta}_i^* &= \frac{\dot{\theta}_i}{\omega}
\end{split}
\end{align}

The variables $\theta_i$ and $\Theta$ are already non-dimensional because radians are dimensionless. Therefore, the function that defines the angle of the first filament is already dimensionless. However, the function that defines the angular velocity is not. This function can be non-dimensionalized as

\begin{align}
\begin{split}
\dot{\theta}_1^* &= - \Theta \frac{\omega}{\omega} \sin \omega t = - \Theta \sin t^*
\end{split}
\end{align}

We can also write
\begin{align}
\begin{split}
\dot{x}_{2}^* &= \frac{1}{N} \Theta \sin t^* \sin \theta_1\\
\dot{y}_{2}^* &= - \frac{1}{N} \Theta \sin t^* \cos \theta_1
\end{split}
\end{align}

We can also introduce two new dimensionless groups: the sperm number ($Sp$) and the anisotropy ratio ($\gamma$).
\begin{align}
\begin{split}
Sp &= L(\frac{\zeta_{\bot} \omega}{A})^\frac{1}{4} \\
\gamma &= \frac{\zeta_{\bot}}{\zeta_{\parallel}}
\end{split}
\end{align}

In the sperm number, $A$ represents a characteristic bending stiffness $EI$. The sperm number indicates the relative influence of viscous and bending forces, while the anisotropy ratio represents the ratio of parallel to perpendicular drag forces.

\newpage

\subsection{Non-Dimensional Force Components}
We can formulate the components of hydrodynamic force in the $\uvec{e}_x$ and $\uvec{e}_y$ directions. The components $F_{i,x}^h$ and $F_{i,y}^h$ can be represented as
\begin{align}
\begin{split}
&F_{i,x}^h = (F_{i,x,\dot{x}}^h) \dot{x}_i + (F_{i,x,\dot{y}}^h) \dot{y}_i + (F_{i,x,\dot{\theta}}^h) \dot{\theta}_i \\
&F_{i,y}^h = (F_{i,y,\dot{x}}^h) \dot{x}_i + (F_{i,y,\dot{y}}^h) \dot{y}_i + (F_{i,y,\dot{\theta}}^h) \dot{\theta}_i \\
\end{split}
\end{align}

\begin{align*}
&F_{i,x,\dot{x}}^h = - l (\zeta_{\bot} \sin^2 \theta_i + \zeta_{\parallel} \cos^2 \theta_i) & &F_{i,y,\dot{x}}^h =\frac{1}{2} (\zeta_{\bot} - \zeta_{\parallel} ) \sin 2 \theta_i \\
&F_{i,x,\dot{y}}^h = \frac{1}{2} (\zeta_{\bot} - \zeta_{\parallel} ) \sin 2 \theta_i & &F_{i,y,\dot{y}}^h =  - l (\zeta_{\bot} \cos^2 \theta_i + \zeta_{\parallel} \sin^2 \theta_i) \\
&F_{i,x,\dot{\theta}}^h = \frac{1}{2} l^2 \zeta_{\bot} \sin \theta_i & &F_{i,y,\dot{\theta}}^h = - \frac{1}{2} l^2 \zeta_{\bot} \cos \theta_i
\end{align*}

The length of each filament is $l = L/N$, where $L$ is the length of all of the individual filaments combined. The length $l$ can be non-dimensionalized using $L$ to give a non-dimensional length of $\frac{1}{N}$.

We can non-dimensionalize the resistive force theory coefficients by diving the equations by $\zeta_{\bot}$. Then we can substitute the anisotropy ratio, $\gamma = \frac{\zeta_{\bot}}{\zeta_{\parallel}}$, into the equations. These steps give the following non-dimensional force coefficients:

\begin{align*}
&F_{i,x,\dot{x}}^* = - \frac{1}{N} ( \sin^2 \theta_i + \frac{1}{\gamma} \cos^2 \theta_i) & &F_{i,y,\dot{x}}^* = \frac{1}{2N} (1 - \frac{1}{\gamma} ) \sin 2 \theta_i \\
&F_{i,x,\dot{y}}^* = \frac{1}{2N} (1 - \frac{1}{\gamma} ) \sin 2 \theta_i & &F_{i,y,\dot{y}}^* =  - \frac{1}{N} ( \cos^2 \theta_i + \frac{1}{\gamma} \sin^2 \theta_i) \\
&F_{i,x,\dot{\theta}}^* = \frac{1}{2} (\frac{1}{N})^2 \sin \theta_i & &F_{i,y,\dot{\theta}}^* = - \frac{1}{2} (\frac{1}{N})^2 \cos \theta_i
\end{align*}

The non-dimensional components of force $F_{i,x}^*$ and $F_{i,y}^*$ can now be represented as
\begin{align}
\begin{split}
&F_{i,x}^* = (F_{i,x,\dot{x}}^*) \dot{x}_i^* + (F_{i,x,\dot{y}}^*) \dot{y}_i^* + (F_{i,x,\dot{\theta}}^*) \dot{\theta}_i^* \\
&F_{i,y}^* = (F_{i,y,\dot{x}}^*) \dot{x}_i^* + (F_{i,y,\dot{y}}^*) \dot{y}_i^* + (F_{i,y,\dot{\theta}}^*) \dot{\theta}_i^* \\
\end{split}
\end{align}

\newpage

\subsection{Non-Dimensional Torque Components}
We can formulate the non-dimensional torque components in a similar manner. We know that the only non-zero component of torque acting on filament $i$ with respect to position $\mathbf{x}_j$ is in the $\uvec{e}_z$ direction. The torque can be written as
\begin{equation}
\uvec{e}_z \cdot \mathbf{T}_{i,j}^h = (T_{\dot{x}_i}) \dot{x}_i + (T_{\dot{y}_i}) \dot{y}_i + (T_{\dot{\theta}_i}) \dot{\theta}_i
\end{equation}

where the components of torque are
\begin{align}
\begin{split}
&T_{\dot{x}_i} = \frac{1}{2} l^2 \zeta_{\bot} \sin \theta_i + \frac{1}{2} l (\zeta_{\bot} - \zeta_{\parallel}) \sin 2 \theta_i (x_i - x_j) + l (\zeta_{\bot} \sin^2 \theta_i + \zeta_{\parallel} \cos^2 \theta_i)(y_i - y_j) \\
&T_{\dot{y}_i} = - \frac{1}{2} l^2 \zeta_{\bot} \cos \theta_i - \frac{1}{2} l (\zeta_{\bot} - \zeta_{\parallel}) \sin 2 \theta_i (y_i - y_j) - l (\zeta_{\bot}\cos^2 \theta_i + \zeta_{\parallel} \sin^2 \theta_i)(x_i - x_j) \\
&T_{\dot{\theta}_i} = - \frac{1}{3} l^3 \zeta_{\bot} - \frac{1}{2} l^2 \zeta_{\bot} \cos \theta_i (x_i - x_j) - \frac{1}{2} \zeta_{\bot} l^2 \sin \theta_i (y_i - y_j)
\end{split}
\end{align}

The torque components can be non-dimensionalized using $L$, $\gamma$, $x^* = \frac{x}{L}$, and $y^* = \frac{y}{L}$.
\begin{align}
\begin{split}
&T_{\dot{x}_i}^* = \frac{1}{2} (\frac{1}{N})^2 \sin \theta_i + \frac{1}{2N} (1 - \frac{1}{\gamma}) \sin 2 \theta_i (x_i^* - x_j^*) + \frac{1}{N} (1 \sin^2 \theta_i + \frac{1}{\gamma} \cos^2 \theta_i)(y_i^* - y_j^*) \\
&T_{\dot{y}_i}^* = - \frac{1}{2} (\frac{1}{N})^2 \cos \theta_i - \frac{1}{2N} (1 - \frac{1}{\gamma}) \sin 2 \theta_i (y_i^* - y_j^*) - \frac{1}{N} (\cos^2 \theta_i + \frac{1}{\gamma} \sin^2 \theta_i)(x_i^* - x_j^*) \\
&T_{\dot{\theta}_i}^* = - \frac{1}{3} (\frac{1}{N})^3 - \frac{1}{2} (\frac{1}{N})^2 \cos \theta_i (x_i^* - x_j^*) - \frac{1}{2} (\frac{1}{N})^2 \sin \theta_i (y_i^* - y_j^*)
\end{split}
\end{align}

The total non-dimensional torque can be written as
\begin{equation}
\uvec{e}_z \cdot \mathbf{T}_{i,j}^* = (T_{\dot{x}_i}^*) \dot{x}_i^* + (T_{\dot{y}_i}^*) \dot{y}_i^* + (T_{\dot{\theta}_i}^*) \dot{\theta}_i^*
\end{equation}

\newpage

\subsection{Non-Dimensional Kinematic Constraints}
Differentiation of the kinematic constraints gives
\begin{align}
\begin{split}
& \dot{x}_i - \dot{x}_{i+1} - l \dot{\theta}_i \sin \theta_i = 0 \\
& \dot{y}_i - \dot{y}_{i+1} + l \dot{\theta}_i \cos \theta_i = 0
\end{split}
\end{align}

We can render this equation dimensionless by dividing it by $L$ and $\omega$. This gives:

\begin{align}
\begin{split}
& \dot{x}_i^* - \dot{x}_{i+1}^* - \frac{1}{N} \dot{\theta}_i^* \sin \theta_i = 0 \\
& \dot{y}_i^* - \dot{y}_{i+1}^* + \frac{1}{N} \dot{\theta}_i^* \cos \theta_i = 0
\end{split}
\end{align}

\subsection{Non-Dimensional Spring Constant}
The non-dimensional sperm number includes the characteristic bending stiffness, $A_0$, in its denominator. The sperm number is used to create a non-dimensional spring constant. We know that $l = L/N$, so we can write
\begin{equation}
k_t = \frac{A_0}{l} = \frac{A_0}{L/N} = \frac{A_0 N}{L}
\end{equation}

We can also write
\begin{equation}
Sp^4 = L^4 \frac{\zeta_{\bot} \omega}{A_0}
\end{equation}

We can perform dimensional analysis on the resistive force theory coefficients. We know that the dimensions of force are $[\frac{ML}{T^2}]$. We also know that $F^h \propto l^2 \omega \zeta$, where the dimensions of $l^2 \omega$ are $[\frac{L^2}{T}]$. Therefore

\begin{equation}
\zeta = [\frac{ML}{T^2} \frac{T}{L^2} ] = [\frac{M}{LT}]
\end{equation}

We also know that the dimensions of the following variables

\begin{align}
\begin{split}
A_0 &= EI = [\frac{M}{LT^2} L^4] = [\frac{ML^3}{T^2}] \\
\frac{\zeta_{\bot} \omega}{A_0} &= [\frac{M}{LT} \frac{1}{T} \frac{T^2}{ML^3}] = [\frac{1}{L^4}] \\
Sp &= L (\frac{\zeta_{\bot} \omega}{A_0})^{\frac{1}{4}} = [L (\frac{1}{L^4})^{\frac{1}{4}}] = [1] \\
k_t &= \frac{NA_0}{L} = [\frac{ML^3}{T^2} \frac{1}{L}] = [\frac{ML^2}{T^2}]
\end{split}
\end{align}

Our goal is to render $k_t$ dimensionless. We can divide $k_t$ by $L^3 \zeta_{\bot} \omega$ to get a non-dimensional spring constant.
\begin{align}
\begin{split}
\zeta_{\bot} \omega &= [\frac{M}{LT} \frac{1}{T}] = [\frac{M}{LT^2}] \\
L^3 \zeta_{\bot} \omega &= [L^3 \frac{M}{LT^2}] = [\frac{ML^2}{T^2}] \\
\frac{k_t}{L^3 \zeta_{\bot} \omega} &= \frac{NA_0}{L} \frac{1}{L^3 \zeta_{\bot} \omega} = \frac{NA_0}{L^4 \zeta_{\bot} \omega} = [\frac{ML^2}{T^2} \frac{T^2}{ML^2}] = [1]
\end{split}
\end{align}

We use the sperm number to write
\begin{equation}
k^* = \frac{NA_0}{L^4 \zeta_{\bot} \omega} = \frac{N}{Sp^4}
\end{equation}

What if the bending stiffness of the filament in question is not equal to the characteristic bending stiffness $A_0$? Each of the filaments could have a different bending stiffness, denoted by $A_i$. Their equivalent torsional spring constant is
\begin{equation}
k_{t,i} = \frac{A_i N}{L}
\end{equation}

We create a new dimensionless variable $C_i = \frac{A_i}{A_0}$, which allows us to write
\begin{equation}
k_i^* = \frac{A_i N}{L} \frac{1}{L^3 \zeta_{\bot} \omega} = \frac{A_i N}{L^4 \zeta_{\bot} \omega} = C_i \frac{A_0 N}{L^4 \zeta_{\bot} \omega} = \frac{C_i N}{Sp^4}
\end{equation}

\newpage
\section{MATRIX FORMULATION}
Previously non-dimensional variables were denoted with a $*$ superscript. From this point forward, non-dimensional variables will be written without the $*$ superscript. We will formulate our problem as $\mathbf{A} \mathbf{\dot{X}} = \mathbf{b}$, where

\[\mathbf{A} = \begin{bmatrix}
T_{\dot{x}}&T_{\dot{y}}&T_{\dot{\theta}}\\
C_{x,\dot{x}}&\mathbf{0}&C_{x,\dot{\theta}}\\
\mathbf{0}&C_{y,\dot{y}}&C_{y,\dot{\theta}}
\end{bmatrix}
\text{, }
\mathbf{\dot{X}} = \begin{bmatrix}
\dot{x}_3\\
\dot{x}_4\\
\vdots \\
\dot{x}_{N+1}\\
\dot{y}_3\\
\dot{y}_4\\
\vdots \\
\dot{y}_{N+1}\\
\dot{\theta}_2\\
\dot{\theta}_3\\
\vdots \\
\dot{\theta}_N
\end{bmatrix}
\text{ and }
\mathbf{b} = \begin{bmatrix}
k_i (\theta_i - \theta_{i-1})\\
\mathbf{0}\\
\mathbf{0}
\end{bmatrix}
\]

Each of $T_{\dot{x}}$, $T_{\dot{y}}$, $T_{\dot{\theta}}$, $C_{x,\dot{x}}$, $C_{y,\dot{y}}$, $C_{x,\dot{\theta}}$, and $C_{y,\dot{\theta}}$ is an $N-1$ by $N-1$ matrix. This gives $3N-3$ equations and $3N-3$ unknowns, meaning that our system is solvable.

The variables $\dot{x}_1$ and $\dot{y}_1$ are not included in this system since they are known to be equal to zero at all points in time. We also know that $\dot{x}_2 = l \Theta \sin t \sin \theta_1$ and that $\dot{y}_2 = - l \Theta \sin t \cos \theta_1$, based on the kinematic constraints. It is also known that $\theta_1 = \Theta \cos t$ and that $\dot{\theta}_1 = - \Theta \cos t$.
\newpage
\subsection{Formulation of Torque Matrices}
Each of the components of torque ($T_{x,\dot{x}}$, $T_{y,\dot{y}}$, and $T_{\theta,\dot{\theta}}$) are functions of $x_i$, $x_j$, $y_i$, $y_j$, and $\theta_i$. The $j$th components indicate the points about which the torque balance is taken. The torque acting on filament $i$, taken about point $j$, is given by
\begin{equation}
\mathbf{T}_{i,j} = (T_{\dot{x}_i}\dot{x}+T_{\dot{y}_i}\dot{y}+T_{\dot{\theta}_i}\dot{\theta})
\end{equation}

Using the known values of $\dot{x}_2$ and $\dot{y}_2$, as well as the equations for the non-dimensional torque components, we find
\begin{align*}
(T_{2,2,\dot{x}}) (\dot{x}_2) &= (\frac{1}{2} (\frac{1}{N})^2 \sin \theta_2)(\frac{1}{N} \Theta \sin t \sin \theta_1) = \frac{1}{2}(\frac{1}{N})^3 \Theta \sin \theta_1 \sin \theta_2 \sin t\\
(T_{2,2,\dot{y}}) (\dot{y}_2) &= (- \frac{1}{2} (\frac{1}{N})^2 \cos \theta_2)(- \frac{1}{N} \Theta \sin t \cos \theta_1) = \frac{1}{2}(\frac{1}{N})^3 \Theta \cos \theta_1 \cos \theta_2 \sin t\\
\end{align*}

Neither $\dot{x}_2$ nor $\dot{y}_2$ appear in  $\mathbf{\dot{X}}$, so we move the above equations to the right hand side of the system. The $N-1$ by $N-1$ torque matrices as follows:

\[
T_{\dot{x}_i}\mathbf{\dot{X}}_{\dot{x}} = \begin{bmatrix}
T_{3,2,\dot{x}}&T_{4,2,\dot{x}}&T_{5,2,\dot{x}}&\hdots&T_{N-2,2,\dot{x}}&T_{N-1,2,\dot{x}}&T_{N,2,\dot{x}}&0\\
T_{3,3,\dot{x}}&T_{4,3,\dot{x}}&T_{5,3,\dot{x}}&\hdots&T_{N-2,3,\dot{x}}&T_{N-1,3,\dot{x}}&T_{N,3,\dot{x}}&0\\
0&T_{4,4,\dot{x}}&T_{5,4,\dot{x}}&\hdots&T_{N-2,4,\dot{x}}&T_{N-1,4,\dot{x}}&T_{N,4,\dot{x}}&0\\
0&0&T_{5,5,\dot{x}}&\hdots&T_{N-2,5,\dot{x}}&T_{N-1,5,\dot{x}}&T_{N,5,\dot{x}}&0\\
\vdots\\
0&0&0&\hdots&0&T_{N-1,N-1,\dot{x}}&T_{N,N-1,\dot{x}}&0\\
0&0&0&\hdots&0&0&T_{N,N,\dot{x}}&0
\end{bmatrix}
\begin{bmatrix}
\dot{x}_3\\
\dot{x}_4\\
\dot{x}_5\\
\vdots\\
\dot{x}_{N-1}\\
\dot{x}_{N}\\
\dot{x}_{N+1}
\end{bmatrix}
\]

\[
T_{\dot{y}_i}\mathbf{\dot{X}}_{\dot{y}} = \begin{bmatrix}
T_{3,2,\dot{y}}&T_{4,2,\dot{y}}&T_{5,2,\dot{y}}&\hdots&T_{N-2,2,\dot{y}}&T_{N-1,2,\dot{y}}&T_{N,2,\dot{y}}&0\\
T_{3,3,\dot{y}}&T_{4,3,\dot{y}}&T_{5,3,\dot{y}}&\hdots&T_{N-2,3,\dot{y}}&T_{N-1,3,\dot{y}}&T_{N,3,\dot{y}}&0\\
0&T_{4,4,\dot{y}}&T_{5,4,\dot{y}}&\hdots&T_{N-2,4,\dot{y}}&T_{N-1,4,\dot{y}}&T_{N,4,\dot{y}}&0\\
0&0&T_{5,5,\dot{y}}&\hdots&T_{N-2,5,\dot{y}}&T_{N-1,5,\dot{y}}&T_{N,5,\dot{y}}&0\\
\vdots\\
0&0&0&\hdots&0&T_{N-1,N-1,\dot{y}}&T_{N,N-1,\dot{y}}&0\\
0&0&0&\hdots&0&0&T_{N,N,\dot{y}}&0
\end{bmatrix}
\begin{bmatrix}
\dot{y}_3\\
\dot{y}_4\\
\dot{y}_5\\
\vdots\\
\dot{y}_{N-1}\\
\dot{y}_{N}\\
\dot{y}_{N+1}
\end{bmatrix}
\]

\[
T_{\dot{\theta}_i}\mathbf{\dot{X}}_{\dot{\theta}} = \begin{bmatrix}
T_{2,2,\dot{\theta}}&T_{3,2,\dot{\theta}}&T_{4,2,\dot{\theta}}&\hdots&T_{N-2,2,\dot{\theta}}&T_{N-1,2,\dot{\theta}}&T_{N,2,\dot{\theta}}\\
0&T_{3,3,\dot{\theta}}&T_{4,3,\dot{\theta}}&\hdots&T_{N-2,3,\dot{\theta}}&T_{N-1,3,\dot{\theta}}&T_{N,3,\dot{\theta}}\\
0&0&T_{4,4,\dot{\theta}}&\hdots&T_{N-2,4,\dot{\theta}}&T_{N-1,4,\dot{\theta}}&T_{N,4,\dot{\theta}}\\
\vdots\\
0&0&0&\hdots&0&T_{N-1,N-1,\dot{\theta}}&T_{N,N-1,\dot{\theta}}\\
0&0&0&\hdots&0&0&T_{N,N,\dot{\theta}}
\end{bmatrix}
\begin{bmatrix}
\dot{\theta}_2\\
\dot{\theta}_3\\
\dot{\theta}_4\\
\vdots\\
\dot{\theta}_{N-1}\\
\dot{\theta}_{N}
\end{bmatrix}
\]

\newpage

\subsection{Formulation of Kinematic Constraint Matrices}
The $N - 1$ by $N - 1$ kinematic constraint matrices ($C_{x,\dot{x}}$, $C_{x,\dot{\theta}}$, $C_{y,\dot{y}}$, and $C_{y,\dot{\theta}}$) are as follows:

\[
(C_{x,\dot{x}}) \mathbf{\dot{X}}_{\dot{x}} = \begin{bmatrix}
-1&0&.&.&\hdots&.&.&.&.\\
+1&-1&0&.&\hdots&.&.&.&.\\
0&+1&-1&0&\hdots&.&.&.&.\\
\vdots \\
.&.&.&.&\hdots&0&+1&-1&0\\
.&.&.&.&\hdots&.&0&+1&-1\\
\end{bmatrix}
\begin{bmatrix}
\dot{x}_3\\
\dot{x}_4\\
\dot{x}_5\\
\vdots\\
\dot{x}_N\\
\dot{x}_{N+1}
\end{bmatrix}
\]

\[
(C_{x,\dot{\theta}}) \mathbf{\dot{X}}_{\dot{\theta}} = - \frac{1}{N} \begin{bmatrix}
\sin\theta_1&0&.&.&.& \hdots &0\\
0&\sin\theta_2&0&.&.& \hdots &0\\
\vdots \\
.&.&.&.\hdots &0&\sin\theta_{N-1}&0\\
.&.&.&.\hdots &.&0&\sin\theta_N\\
\end{bmatrix}
\begin{bmatrix}
\dot{\theta}_2\\
\dot{\theta}_3\\
\vdots\\
\dot{\theta}_{N-1}\\
\dot{\theta}_N\\
\end{bmatrix}
\]

\[
(C_{y,\dot{y}}) \mathbf{\dot{X}}_{\dot{y}} = \begin{bmatrix}
-1&0&.&.&\hdots&.&.&.&.\\
+1&-1&0&.&\hdots&.&.&.&.\\
0&+1&-1&0&\hdots&.&.&.&.\\
\vdots \\
.&.&.&.&\hdots&0&+1&-1&0\\
.&.&.&.&\hdots&.&0&+1&-1\\
\end{bmatrix}
\begin{bmatrix}
\dot{y}_3\\
\dot{y}_4\\
\dot{y}_5\\
\vdots\\
\dot{y}_N\\
\dot{y}_{N+1}
\end{bmatrix}
\]

\[
(C_{y,\dot{\theta}}) \mathbf{\dot{X}}_{\dot{\theta}} = + \frac{1}{N} \begin{bmatrix}
\cos\theta_1&0&.&.&.& \hdots &0\\
0&\cos\theta_2&0&.&.& \hdots &0\\
\vdots \\
.&.&.&.\hdots &0&\cos\theta_{N-1}&0\\
.&.&.&.\hdots &.&0&\cos\theta_N\\
\end{bmatrix}
\begin{bmatrix}
\dot{\theta}_2\\
\dot{\theta}_3\\
\vdots\\
\dot{\theta}_{N-1}\\
\dot{\theta}_N\\
\end{bmatrix}
\]

\newpage

\subsection{Formulation of RHS Matrix}
The right hand column vector has $3N-3$ entries. The first $N-1$ entries correspond to the torque balance, the second set of $N-1$ entries correspond to the kinematic constraints in the $x$ axis, and the last $N-1$ entries correspond to the kinematic constraints in the $y$ axis. All entries for the kinematic constraints are zero, except for the entries in the $2(N-1)$ and the $3(N-1)$ positions, which are equal to $-\dot{x}_2$ and $-\dot{y}_2$ respectively. All entries in the first set of $N-1$ entries are $k_i(\theta_i - \theta_{i-1})$, except for the first entry.

We can write the first entry as
\begin{align*}
\mathbf{b}[0] &= k_2 (\theta_2 - \theta_1) - T_{2,2,\dot{x}}[\dot{x}_2] - T_{2,2,\dot{y}}[\dot{y}_2] \\
&= k_2 (\theta_2 - \theta_1) -  \frac{1}{2}(\frac{1}{N})^3 \Theta \sin \theta_1 \sin \theta_2 \sin t - \frac{1}{2}(\frac{1}{N})^3 \Theta \cos \theta_1 \cos \theta_2 \sin t
\end{align*}

Therefore, the RHS matrix is:

\[
\mathbf{b} = \begin{bmatrix}
k_2 (\theta_2 - \theta_1) - T_{2,2,\dot{x}}(\dot{x}_2) - T_{2,2,\dot{y}}(\dot{y}_2)\\
k_3 (\theta_3 - \theta_2)\\
k_4 (\theta_4 - \theta_3)\\
\vdots \\
k_N (\theta_N - \theta_{N-1})\\
- \frac{1}{N} \Theta \sin t \sin \theta_1\\
0_2 \\
0_3 \\
\vdots \\
0_N \\
\frac{1}{N} \Theta \sin t \cos \theta_1\\
0_2 \\
0_3 \\
\vdots \\
0_N \\
\end{bmatrix} 
\]

\section{Validation}
The results of the simulation must be validated to confirm they make sense. A simple initial test is to ensure the kinematic constraints are satisfied. This means that the distance between each filament is $1/N$. We can find this distance using the following equation:
\begin{equation}
\text{Distance: } = \sqrt{(x_i - x_{i+1})^2 + (y_i - y_{i+1})^2}
\end{equation}

\end{document}
